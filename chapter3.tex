% !TeX root = document.tex
% !TeX encoding = UTF-8 Unicode

\chapter{Configuração de Hardware}%
\label{chapter:hardware-configuration}

Na seção \textit{Configuração de Hardware} é feito a configuração do driver, que
controla o hardware. Esse driver pode ser: snap7, Arduino, GenericTCPIO, Dummy,
Oven. Para mais informações sobre driver, acesse: <...> Além disso, é feito as
configurações das portas de entradas ou saídas do sistema. Entre essas
configurações está:

\begin{itemize}
    \item Nome da porta
    \item Alias da porta (nome que a variavel dentro do programa)
    \item Tipo de porta: analógica ou digital
    \item Valor desligado (valor que deve assumir quando o sistema está desligado)
\end{itemize}

O Alias da porta é o nome dado para a porta no aplicativo. O tipo de porta
varia de acordo com a escolha do driver. Para incluir uma porta, deve-se clicar
no opção “Adicionar” no tópico de portas dentro da seção configuração de
hardware. Para remover, é necessário clicar na opção “Remover” localizada na
frente da porta que deseja ser excluída. O Valor desligado é o valor que a
porta deve assumir quando o sistema está desativado.

No tópico de Calibração dessa seção é feita a calibração das portas. A porta
recebe o nome definido em Alias da porta. O "valor" da porta não é o mesmo,
logo, precisa de um novo Alias. Os valores inseridos na Calibração Estática
devem ser calculados por meio de experimentos e testes nas portas.

No tópico Intertravamento é possível realizar prevenção no sistema para evitar
falhas. Exemplo disso é o Intertravamento em um sistema de tanques
comunicantes, caso o nível de água ultrapasse 70 centímetros, o teste é parado
e a variável SystemOnOff assume valor zero, desligando o sistema.
	
No final da seção de Configuração de Hardware há opções de exportar, importar
ou reinicializar configurações de hardware para ou de um arquivo. Ainda há a
opção de reinicializar o formulário. Para ativar uma configuração nova é
necessário clicar na opção “Aplicar” no final da seção.
