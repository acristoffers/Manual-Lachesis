% !TeX root = document.tex
% !TeX encoding = UTF-8 Unicode

\chapter{Introdução}%
\label{chapter:introduction}

A plataforma de controle foi desenvolvida para auxiliar no ensino de controle e
de análise de sistemas. Ela foi desenvolvida de forma que o usuário possa
utilizar a linguagem Python para escrever seus controladores.

Os dois principais objetivos da plataforma é rodar em um \textit{hardware} com
baixo poder processamento, como o \textit{Raspberry Pi} e exibir gráficos em
tempo real. Para isso ela foi divida em dois aplicativos. O primeiro, denominado
\textbf{moirai}, deve ser instalado na máquina que faz interface com o
\textit{hardware} e funciona como um servidor. O segundo, \textbf{Lachesis},
funciona como cliente e é a inteface gráfica que será utilizada pelo usuário
para gerenciar os teste e realizar as configurações.

% \FloatBarrier
% \begin{figure}[ht!]
%     \centering
%     \includegraphics[width=8.5cm]{imgs/compiled_vs_interpreted}
%     \caption[Compilado vs Interpretado]{Compilado vs Interpretado}
%     \label{fig:compiled-vs-interpreted}
% \end{figure}
% \FloatBarrier
