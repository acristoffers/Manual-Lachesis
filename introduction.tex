% !TeX root = document.tex
% !TeX encoding = UTF-8 Unicode

\chapter{Introdução}%
\label{chapter:introduction}

A plataforma de controle foi desenvolvida para auxiliar no ensino de controle e
de análise de sistemas. Ela foi desenvolvida de forma que o usuário possa
utilizar a linguagem \textit{Python} para escrever seus controladores.

Os dois principais objetivos da plataforma é rodar em um \textit{hardware} com
baixo poder processamento, como o \textit{Raspberry Pi} e exibir gráficos em
tempo real. Para isso ela foi dividida em dois aplicativos. O primeiro, denominado
\textbf{moirai}, deve ser instalado na máquina que faz interface com o
\textit{hardware} e funciona como um servidor. O segundo, \textbf{Lachesis},
funciona como cliente e é a inteface gráfica que será utilizada pelo usuário
para gerenciar os teste e realizar configurações.

\section{Instalação}
\label{sec:installation}

O \textbf{moirai} é escrito em \textit{Python} e requer que o mesmo esteja
instalado na máquina. Há várias maneiras de se obter o interpretador
\textit{Python}. Ele pode ser baixado diretamente do site
\href{https://www.python.org}{python.org}, instalado pelo gerenciador de pacotes
do seu sistema operacional (apt, yum, brew, etc), instalado por gerenciadores de
versão (\href{https://github.com/pyenv/pyenv}{pyenv}) ou por uma suíte
(\href{https://www.continuum.io/downloads}{Anaconda}).

Este último é necessário na plataforma Windows já que não há binários para esta
plataforma no repositório \textit{PyPi} e a compilação é extremamente difícil
devido a falta de suporte das bibliotecas usadas pela suíte \textit{SciPy}.

Seguem as maneiras mais fáceis de se instalar no Windows, Ubuntu e macOS:

\begin{itemize}
\item \textbf{Windows}
        Baixe e instale o \textit{bundle} Anaconda
        (\url{https://www.continuum.io/downloads}, também disponível para outras
        plataformas).

\item \textbf{Ubuntu}
        \mintinline{bash}{sudo apt install python3-scipy python3-matplotlib}

\item \textbf{macOS}
        Instale o Homebrew: \url{https://brew.sh}\\
        \mintinline{bash}{brew install python3}\\
        \mintinline{bash}{pip3 install numpy matplotlib}
\end{itemize}

Após instalado o \textit{Python}, deve-se instalar o \textbf{moirai} utilizando
o gerenciador de pacotes \textit{pip}. Para isso basta executar em um terminal:

\mintinline{bash}{pip3 install moirai}

O aplicativo depende de um banco de dados (\textit{MongoDB} ou \textit{MySQL})
e, caso deseje-se utilizar o PLC da Siemens, da biblioteca \textit{Snap7}. A
instalação desses foi inserida como opção no aplicativo, e pode-se executá-la
através do comando:

\mintinline{bash}{moirai --install --sudo}

Os aplicativos serão baixados e instalados conforme necessário. Caso a
instalação falhe, uma mensagem de erro será exibida mostrando qual software não
pôde ser instalado e pedindo que a instalação manual seja realizada.

Uma vez que as dependências tenham sido instaladas, deve-se configurar uma senha
para o aplicativo, que será necessária para realizar a conexão pelo
\textbf{Lachesis}. Para isso execute

\mintinline{bash}{moirai --set-password=1234}

substituindo 1234 pela senha desejada.

Para executar o aplicativo, simplesmente digite \mintinline{bash}{moirai} em um
terminal.

% \FloatBarrier
% \begin{figure}[ht!]
%     \centering
%     \includegraphics[width=8.5cm]{imgs/compiled_vs_interpreted}
%     \caption[Compilado vs Interpretado]{Compilado vs Interpretado}
%     \label{fig:compiled-vs-interpreted}
% \end{figure}
% \FloatBarrier
