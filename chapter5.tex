% !TeX root = document.tex
% !TeX encoding = UTF-8 Unicode

\chapter{Controle do Sistema}%
\label{chapter:controle-do-sistema}

Na seção Controle do Sistema é adicionado códigos que irão atuar como
controladores da planta. Para adicionar um controlador, é necessário clicar em
“Adicionar” e configurar um formulário. Nesse formulário deve ser especificado o
nome do controlador, o tempo de amostragem (em segundos), o tempo total de
execução (em segundos) e os sensores que devem ser lidos. Além disso, há a
necessidade de desenvolver o código que será executado pelo controlador. Esse
código é divido em 3 partes: Antes, Controlador e Depois. É no código Antes que
deve ser calculados variáveis que permaneceram constantes durante o teste, como
kp, ki e kd. O Controlador deve ser inserido o código que será executado nos
intervalos de amostragens. O Depois é o código que será executado depois do
Controlador, mesmo se algo der errado.

Na tela da seção Controle do Sistema é listado todas as configurações de
controladores. Para executar um controlador, deve-se clicar em “Executar”. Para
editar um teste já definido, deve-se clicar em Editar. Para fazer uma cópia,
deve-se clicar em Clonar. 
