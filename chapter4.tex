% !TeX root = document.tex
% !TeX encoding = UTF-8 Unicode

\chapter{Resposta do Sistema}%
\label{chapter:reposta-do-sistema}

Na seção Resposta do Sistema é feito a configurações de testes para malha
aberta.  Para isso, deve adicionar um teste e configurar o formulário. No
formulário, há opções de tipos de sinal, que variam entre: Degrau, Impulso,
Pulso largo, Escada, Forma livre. 

No sinal de Degrau, é possível configurar os dados da curva por meio da
alteração dos valores de V0, ∆V, T0 e T1. Além disso, é preciso escolher as
entradas em que se deseja controlar e saídas a se registrar. A Taxa de registro
é o tempo em segundos em que o aplicativo deve armazenar os dados. As Saídas
fixas são portas que, caso adicionadas, vão permanecer no mesmo valor no
decorrer do teste. No tópico de Saídas após os testes é possível definir um
valor que uma porta deve assumir quando o teste terminar. Para salvar o teste, é
necessário clicar no botão Salvar.

No caso do sinal de Impulso, na configuração dos Dados da curva, ao invés de
definir o ∆V, é configurado o ∆T. Quando o sinal for Pulso largo, deve ser
feito, diferentemente dos outros tipos de sinais, a definição dos valores de
∆V1, ∆V2 e T2. Caso o tipo do sinal seja Escada, é preciso configurar V1, n
(número de degraus) e t (tempo).

Além desses tipos de sinais, é possível criar um sinal com forma livre. Para
isso, há a possibilidade de importar um modelo já pronto. Outra forma de criar
um sinal é a definição dos pontos por meio da configuração do eixo x (tempo em
segundos) ou do eixo y (valor).

Na tela dessa seção é listado todas as configurações de testes. Para executar um
teste, deve-se clicar em Executar. Para editar um teste já definido, deve-se
clicar em Editar. Para fazer uma cópia, deve-se clicar em Clonar. 
